\documentclass{article}
\usepackage[brazil]{babel}
\usepackage{amsmath}
\usepackage{amssymb}
\usepackage{amsthm}
\usepackage{geometry}
\usepackage{indentfirst}
\author{
   \textbf{Luisa Costa Gualano Borsa} (Candidata)\\
    \textbf{Vinicius de Oliveira Rodrigues} (Orientador)\\
}
\title{\textbf{Forcing, Combinatória Infinita e Aplicações}}
\date{Maio de 2025}
\begin{document}
\maketitle
\begin{abstract}
    Este projeto foca no estudo técnica de forcing, tema central da teoria dos conjuntos com aplicações nesta e em outras áreas da matemática. Serão abordados seus fundamentos, pré-requisitos e aplicações clássicas em Combinatória Infinita, como a prova da consistência da negação da hipótese do contínuo e do Axioma de Martin.
\end{abstract}



\section{Introdução}
    A Teoria dos Conjuntos, introduzida por Georg Cantor no final do século XIX, estabeleceu um arcabouço teórico rigoroso para a formalização da Matemática e a compreensão de estruturas infinitas. No início do século passado, essa teoria foi formalizada em lógica de primeira ordem, sendo ZF as mais conhecidas e utilizadas axiomatizações a teoria ZF (de Zermelo-Fraenkel) e, principalmente, sua extensão ZFC (que inclui o Axioma da Escolha).

    Uma das mais conhecidas questões clássicas que surgiram deste estudo é a validade da \emph{Hipótese do Contínuo}.
    Formulada por Cantor em 1878 \cite{cantor1878beitrag}, a Hipótese do Contínuo afirma que a cardinalidade do conjunto dos números reais é a menor possível dentre as cardinalidades não enumeráveis -- em símbolos, que $\mathfrak c=\aleph_1$.
    Tal questão adquiriu grande notoriedade, figurando como a primeira questão da conhecida lista de problemas de Hilbert de 1900 \cite{hilbert1900mathematische}.

    Na década de 1940, Gödel mostrou que a Hipótese do Contínuo é válida no modelo interno dos conjuntos construtíveis, $\mathbf{L}$, assim mostrando que, se ZF é consistente, então ZFC+CH também é consistente.
    Equivalentemente, Gödel mostrou que, se ZF é consistente, então a Hipótese do Contínuo não pode ser refutada a partir dos axiomas de ZFC.
    
    Após o trabalho de Gödel, seguiu em aberto a questão sobre se CH pode ser provado a partir dos axiomas de ZFC.
    Tal questão foi respondida negativamente apenas em 1963, quando Paul Cohen \cite{cohen1963independence} introduziu a técnica de \emph{forcing} para produzir modelos de teoria dos conjuntos em que a Hipótese do Contínuo é falsa.

    Forcing se transformou rapidamente em uma das mais importantes ferramentas em Teoria dos conjuntos.
    Em 1971, Solovay e Tennenbaum \cite{solovay1971iterated} adaptaram esta técnica para aplica-la de forma iterada a fim de construir um modelo de ZFC sem árvore de Suslin -- um importante objeto combinatório estudado sistematicamente desde a década de 30 cuja existência em ZFC era uma pergunta em aberto, e relacionado com o problema de classificação da reta real \cite{kurepa1935ensembles}.

    O Axioma de Martin, axioma que captura certos argumentos envolvendo forcing sem fazer uso de modelos, foi introduzido e teve sua consistência provada por Martin e Solovay em \cite{martin1970internal}, impulsionando a popularização da técnica de forcing.

    O estudo das características cardinais do contínuo -- cardinais entre o primeiro cardinal não enumerável, $\aleph_1$ e o contínuo, $\mathfrak c$, que capturam propriedades do contínuo que conjuntos enumeráveis não possuem, também ganhou grande tração, sendo Forcing a principal ferramenta para provar que tais cardinais podem ser arbitrariamente grandes ou pequenos, além de distintos entre si.
    Trabalhos clássicos sobre esse tema são \cite{bartoszynski1993chichon} e \cite{miller1981some}, e surveys mais modernos sobre esse material incluem \cite{blass2009combinatorial} e \cite{douwen1984integers}. Trabalhos crescentes continuam sendo publicados em revistas de ponta (sendo, por exemplo, \cite{goldstern2019chichon} um destaque recente).

    Há inúmeras aplicações de Combinatória Infinita e Forcing em outras áreas da Matemática, como por exemplo em Topologia Geral, conforme fica explícito nos handbooks \cite{kunen1984handbook} e \cite{foreman2009handbook}, com uma ampla gama de resultados envolvendo inclusive aplicações de Topologia Geral em Análise Funcional e Álgebra.

    Apesar do grande número de aplicações importantes e do amplo desenvolvimento da técnica de Forcing ao longo das décadas, forcing continua sendo um tópico denso que requer esforço e dedicação para ser aprendido e plenamente compreendido. Tal tópico não é parte do currículo regular de Matemática no Brasil, nem mesmo em um primeiro curso de Teoria Axiomática dos Conjuntos como o oferecido pelo IME-USP.

    Neste projeto, visamos capacitar a aluna de iniciação científica a trabalhar com técnicas de forcing a partir de um estudo guiado sistemático em que esta adquirirá os pré-requisitos restantes necessários para sua plena compreensão, estudando a seguir sua e suas aplicações básicas clássicas em detalhes (apresentadas com roupagem moderna), seguindo para aplicações mais recentes e complexas.
    Ao final da iniciação científica, a aluna estará preparada para iniciar a leitura de artigos científicos contemporâneos da área, tendo plenas condições de participar de seminários de pós-graduação no instituto e contribuir com discussões, e poderá, futuramente, desenvolver projeto de pesquisa de pós-graduação com potencial de impacto muito maior do que possuiria sem que a iniciação científica fosse realizada.

    \subsection{Apresentação da candidata e do orientador}

    A candidata, Luisa Costa Gualano Bossa, é estudante do Bacharelado em Matemática do IME-USP, tendo ingressado via transferência externa no segundo semestre de 2024, e atualmente cursa o equivalente ao terceiro semestre do curso.
    Na data de início no projeto, estará cursando o equivalente ao quarto semestre.
    A candidata tem obtido excelente desempenho nas disciplinas e demonstra grande interesse em Teoria dos Conjuntos, estando atualmente estudando os capítulos iniciais do livro \cite{hrbacek2017introduction} em formato de pré-iniciação científica junto ao orientador, também demonstrando grande interesse, dedicação e desempenho.

    O orientador, Vinicius de Oliveira Rodrigues, é Professor Doutor do Departamento de Matemática do IME-USP com ampla experiência em Teoria dos Conjuntos e Topologia Geral, tendo realizado estágios no exterior na York University e University of Toronto.
    Utiliza Forcing como ferramenta de pesquisa frequente e fundamental, como por exemplo em \cite{bellini2021forcing}, \cite{carvalho2024products}, \cite{corral2022fin}, \cite{guzman2022maximal} e \cite{rodrigues2021almost}, estando, assim, plenamente apto a orientar este projeto.

    A instituição sede, o IME-USP, concentra o maior núcleo de pesquisa em Teoria dos Conjuntos e Set Theoretic Topology do Brasil, contando com outros docentes especialistas na área, bem como um considerável número de alunos de pós-graduação, sendo este um ambiente adequado para o desenvolvimento do projeto. 

    \section{Objetivos}
    O principal objetivo do projeto é que, ao seu final, a aluna compreenda plenamente a técnica de forcing e suas aplicações clássicas, de modo a capacita-la para iniciar a leitura de artigos contemporâneos da área e participar de seminários de pós-graduação no IME-USP, contribuindo com discussões de pesquisa.

    Os tópicos a serem estudados incluem:

    \begin{itemize}
        \item Teoria dos modelos básica, entendento a noção de estrutura, modelos e os teoremas de Löwenheim-Skolem-Tarski, compacidade, completude e correção.
        \item Pré-requisitos restantes sobre Teoria dos Conjuntos, como um básico sobre a classe $\mathbf{L}$ dos construtíveis e os Teoremas da Reflexão.
        \item A noção de absolutidade em Teoria dos Conjuntos, fórmulas $\Delta_0$.
        \item A teoria geral envolvendo a técnica de Forcing, incluindo as demonstrações dos Teoremas da Verdade e Definibilidade.
        \item O modelo de Cohen para a prova da negação da consistência da Hipótese do contínuo e algumas das suas principais propriedades.
        \item O Axioma de Martin e algumas aplicações. Forcing iterado e a demonstração de sua consistência.
        \item Se houver tempo, outras aplicações em detalhes da técnica de Forcing que sejam de maior interesse da aluna a serem decididos em momento oportuno. Tópicos possíveis envolvem problemas de Topologia Geral, Teoria dos Grafos Infinitos ou Características Cardinais do Contínuo, mas sem se restringir a estes.
    \end{itemize}

    Durante o projeto, a aluna começará a comparecer e participar de seminários de pós-graduação do IME-USP, onde poderá discutir os tópicos estudados e suas aplicações com outros alunos e professores, tendo contato, nestes seminários, com outras aplicações de Forcing e Combinatória Infinita.

\section{Plano de Trabalho, Metodologia e Execução}
    Na etapa pré-projeto, a aluna seguirá o estudo dos capítulos iniciais do livro \cite{hrbacek2017introduction}, de modo a dominar a manipulação básica de ordinais, cardinais, e compreenda induções e recursões transfinitas. A atual grande dedicação, desempenho e interesse da aluna nos atuais seminários de pré-iniciação científica não deixa dúvidas ao orientador de que, ao início do projeto, estes pré-requisitos estarão supridos.

    \subsection{Primeiros seis meses - Pré-requisitos}
    Os primeiros seis meses de projeto consistirão em um estudo direcionado sistemático aos pré-requisitos de Forcing.
    \subsubsection{Teoria dos Modelos básica e absolutidade de fórmulas}
    Iniciaremos com o estudo do livro \cite{kunen2009foundations}, capítulos II.1-II.12, II.15 e II.16. Alguns dos principais resultados a serem estudados incluem:
    \begin{itemize}
        \item Teorema da compacidade: seja $T$ uma teoria de primeira ordem. Se todo subconjunto finito de $T$ possui um modelo, então $T$ possui um modelo.
        \item Teorema de Löwenheim-Skolem-Tarski para baixo: seja $\mathcal L$ uma linguagem de primeira ordem, $M$ uma $\mathcal L$-estrutura e $A\subseteq M$. Então $M$ admite uma subestrutura elementar $N$ tal que $A\subseteq N$ e $|N|\leq |A|+\aleph_0$.
        \item Teorema da correção: seja $T$ uma teoria de primeira ordem e $\phi$ uma sentença. Se $T$ prova $\phi$, então $\phi$ é válida em todo modelo de $T$.
        \item Teorema da completude: seja $T$ uma teoria de primeira ordem. Se $T$ é válida em todo modelo de $T$, então $T$ prova $\phi$.
    \end{itemize}
    \subsubsection{A classe dos conjuntos bem-fundados e Teoremas da Reflexão}
    Após visitar a teoria dos modelos básica, prosseguiremos para o estudo da classe dos conjuntos bem-fundados e de formas mais gerais do Teorema da Recursão a partir do livro \cite{kunen2011set}. Para tanto, serão estudados as seções I.9 e II.1-II.5 deste livro.
    \subsubsection{A classe dos conjuntos construtíveis}
    Após isso, prosseguiremos para o estudo da classe dos conjuntos construtíveis de Gödel, $\mathbf{L}$, e suas propriedades a partir da seção II.6 do livro \cite{kunen2011set}. Tais resultados incluem o fato de que $\mathbf L$ é o menor modelo interno de ZF, e que valem o Axioma da Escolha a Hipótese Generalizada do Contínuo em $\mathbf L$.
    \subsection{Últimos seis meses - Forcing}
    Caso o programa acima seja cumprido em menos de seis meses, a aluna poderá optar, no tempo restante, entre estudar algum dos tópicos acima com maior profundidade, ou assunto relacionado a estes, ou adiantar o início do estudo de Forcing.

    Quando este tiver início, o estudo será baseado no livro \cite{kunen2011set}, capítulos IV e V.

    As seções IV.1 e IV.2 introduz a técnica de forcing e prova seus teoremas mais fundamentais, como o Teorema da Verdade e o Teorema da Definibilidade.
    
    A seção IV.3 estuda em detalhes o modelo de Cohen, provando, em particular, a consistência da negação da Hipótese do Contínuo.

    A seção IV.4 dá ferramentas essenciais para a caracterização combinatórica de certos modelos de forcing, bem como para a iteração do forcing.

    As seções IV.5-IV.8 introduzem resultados adicionais, que serão estudados parcialmente ou completamente a depender do tempo disponível e do direcionamento desejado pela aluna.

    As seções V.1-V.4 introduzem o forcing iterado, seus resultados teóricos mais básicos e a demonstração do Axioma de Martin. Nesta etapa, a depender do tempo disponível e interesse da aluna, poderemos estudar aplicações do Axioma de Martin em Combinatória Infinita ou outras áreas da matemática.

    As seções V.5-V.7 introduzem resultados adicionais, que serão estudados parcialmente a depender do tempo disponível e do direcionamento desejado pela aluna.

    Caso houver tempo, poderão, ao invés de resultados redigidos nas seções IV.5-IV.8 e V.5-V.7, serem estudados artigos científicos modernos que sejam de interesse da aluna naquele momento e que apliquem ou se relacionem com as técnicas aprendidas.

    \subsection{Metodologia e Execução}
    A metodologia de trabalho é a usual em matemática: a aluna lerá os textos indicados e, semanalmente, se encontrará com o orientador para apresentar e discutir os tópicos estudados, tirar dúvidas e planejar o estudo da semana seguinte.

    O orientador indicará, periodicamente, alguns exercícios e teoremas presentes no texto para ter suas resoluções ou demonstrações digitadas pela aluna a fim de serem apresentados no relatório final do projeto.

    O conteúdo a ser estudado pelo projeto é denso e complexo, mas o interesse e desempenho da aluna, demonstrados em disciplinas e nos seminários de pré-iniciação científica, certificam que ela é apta para realizá-lo.
    Além disso, a experiência do orientador com estes textos e com o tema garantem que a aluna terá o suporte necessário para tirar suas dúvidas, de modo que acredita-se que ela terá progresso substancialmente maior e mais rápido e mais profundo do que caso teria se estudasse os textos como auto-didata.

    \subsection{Forma de análise dos resultados}
    O orientador indicará, periodicamente, alguns exercícios e teoremas presentes no texto para ter suas resoluções ou demonstrações digitadas pela aluna a fim de serem apresentados no relatório final do projeto. O progresso da aluna poderá ser avaliado pelo seus relatórios.
    
    Buscaremos a oportunidade de apresentar os resultados estudados em simpósios, seminários ou congressos.

    Espera-se que, ao final, a aluna esteja preparada para iniciar a leitura de artigos científicos contemporâneos da área, tendo plenas condições de participar de seminários de pós-graduação no instituto, e podendo, futuramente, desenvolver projeto de pesquisa de pós-graduação com potencial de impacto muito maior do que possuiria sem que a iniciação científica fosse realizada.
\bibliographystyle{plain}
\bibliography{bibliografia.bib}
\end{document}